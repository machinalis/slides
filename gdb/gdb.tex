% $Header$
\documentclass[xetex]{beamer}
%\usepackage[utf8]{inputenc} % Este paquete permite poner acentos y eñes usando codificación utf-8
\usepackage[spanish]{babel}
\usepackage{listings}
\lstset{language=C,basicstyle=\small \ttfamily,numbers=left}

\usepackage{xunicode} %Unicode extras!
\usepackage{xltxtra}  %Fixes
\usepackage{fontspec} 
\defaultfontfeatures{Mapping=tex-text} 
\setmainfont{OFL Sorts Mill Goudy}
\setmonofont{Inconsolata}
\setsansfont{OFL Sorts Mill Goudy}
\setromanfont{OFL Sorts Mill Goudy}
 
%template
\setbeamertemplate{navigation symbols}{} 
\usebackgroundtemplate{\includegraphics[width=\paperwidth,height=\paperheight]{background}}
\setbeamercolor{normal text}{fg=white} 
\setbeamercolor{alerted text}{fg=white} 
\setbeamercolor{example text}{fg=white} 
\setbeamercolor{structure}{fg=white} 


\title{\includegraphics[width=6em]{archer.pdf}\\GDB}
\author{Felipe Manzano\\ felipe.andres.manzano@gmail.com}
\institute{Machinalis}
\date{06/06/2011}
\subject{Curso en Intel}

\begin{document}
{\usebackgroundtemplate{\includegraphics[width=\paperwidth,height=\paperheight]{machinalis_a}}
\begin{frame}{}\end{frame}
}

\begin{frame}
  \titlepage
\end{frame}


\begin{frame}
\frametitle{Temas - GDB}
\tableofcontents
\end{frame}

\section{Introducción}
\begin{frame}[fragile]{Introducciósen - historia}
\begin{quotation}
El propósito de un depurador como GDB es permitir al usuario ver lo que
está ocurriendo ‘‘dentro’’  de  otro  programa  mientras  que  se  está
ejecutando — o lo que estaba haciendo en el momento que falló.
\end{quotation}
{\hglue 20em del 'man gdb'}
\begin{itemize}
\item GDB fue inicialmente escrito por Richard Stallman en 1986 
\item Es software libre con licencia GNU
\item Actualmente está en la versión 7.2.x
\end{itemize}
\end{frame}

\begin{frame}{Introducción - características}
 GDB puede hacer cuatro tipos de magia para capturar errores en el acto:
\begin{itemize}
\item Ejecutar un programa desde cero, especificando  cualquier  cosa  que  pueda
afectar a su comportamiento
\item Detener la ejecución de un programa al cumplirse ciertas condiciones
especificadas
\item Examinar qué ha pasado en un crash post-mortem
\item Cambiar cualquier detalle de la ejecución de un programa, memoria, registros e incluso ejecutar funciones del programa
\item Puede depurar C, C++, y código nativo entre otros.
\item Robusto y automatizable
\end{itemize}
\end{frame}

\begin{frame}[fragile]{Introducción - Entrando al gdb}
\begin{itemize}
\item GDB se ejecuta simplemente con el comando \verb=gdb=
\item Las formas más usuales de llamar al gdb son:
\begin{itemize}
\item \verb=gdb -exec programa=
\item \verb=gdb -exec programa -core core= donde \verb=core= es un file con un {\it coredump}
\item \verb=gdb -exec programa -core 1234= donde \verb=1234= es el PID
\item \verb=gdb -silent= no imprime la versión ni la licencia
\item \verb=gdb -help= imprime la ayuda para la línea de comandos
\end{itemize}
\item otra interesante es \verb=-x file.txt= donde se le pasa una lista de comandos internos que \verb=gdb= ejecutará al principio.
\end{itemize}

\end{frame}

\begin{frame}[fragile]{Introducción - Una vez adentro del gdb}
\begin{itemize}
\item Una vez adentro GDB presenta un {\it shell} con todos los beneficios de \verb=GNU readline=
\item Usar y abusar del \verb=CTRL+R=(o \verb=ESC-?=) y \verb=TAB=
\item Se sale con \verb=quit= o \verb=CTRL-D= 
\item Se puede llamar al shell directamente desde gdb así: 
\begin{verbatim}
(gdb) shell ls -al
\end{verbatim}
\item Es fácil recompilar sin salir del gdb:
\begin{verbatim}
(gdb) make makeargs
\end{verbatim}
es equivalente a 
\begin{verbatim}
(gdb) shell make makeargs
\end{verbatim}
\item Tiene un \verb=help= y un \verb=apropos=
\end{itemize}
\end{frame}


\begin{frame}[fragile]{Introducción - Compilar para depuración}
\begin{itemize}
  \item La información de depuración se incluye con \verb=-g= o \verb=-ggdb=
\begin{verbatim}
$ gcc -ggdb -o test.dbg test.c
\end{verbatim}
  \item Para borrar esta información antes de un release:
\begin{verbatim}
$ cp test test.dbg
$ strip test
\end{verbatim}
  \item Nos quedamos con una copia de la versión que contiene la información de depuración
  \item Los {\it coredump} generados con la versión de release podrán usarse con la versión de depuración
  \item Para analizar bugs privados es mejor usar una versión sin optimizaciones:
\begin{verbatim}
$ gcc -ggdb -O0 -o test.dbg test.c
\end{verbatim}
  \item Para incluir información sobre los macros usar algo así:
\begin{verbatim}
$ gcc -gdwarf-2 -g3 sample.c -o sample. 
\end{verbatim}
\end{itemize}

\end{frame}

\section{Primera sesión de gdb}
\begin{frame}[fragile]{Primera sesión de gdb - Un ejemplito (dataentry)}
\begin{lstlisting}
#include<stdio.h>
#include<assert.h>
int
main (int argc, char *argv[])
{
  int edad;
  char nombre[10];
  printf ("Ingrese su nombre: ");
  scanf ("%100s", nombre);
  printf ("Ingrese su edad: ");
  scanf ("%d", &edad);
  printf ("Hola %s, ud tiene %d pirulos!!\n",\
                                nombre, edad);
  return 0;
}
\end{lstlisting}
\end{frame}

\begin{frame}[fragile]{Primera sesión de gdb - comandos útiles (1/2)}
\begin{itemize}
\item Se compila así:
\begin{verbatim}
$ gcc -O0 -g dataentry.c -o dataentry
\end{verbatim}

\item El depurador se ejecuta así: 
\begin{verbatim}
$ gdb dataentry
\end{verbatim}
\item Comandos útiles: 
    \begin{itemize}
        \item \verb=start=
        \item \verb=print edad=
        \item \verb=print nombre=
        \item \verb=next=
        \item \verb=shell man scanf=
        \item \verb=quit= 
    \end{itemize}
\item Maximiliano(18) y Segismundo(65) han reportado problemas... debug! 
\end{itemize}
\end{frame}

\begin{frame}[fragile]{Primera sesión de gdb - más comandos útiles}
\begin{itemize}
\item GDB usa expresiones C-like para hacer referencia a los datos
\item El comando \verb=print EXPR= espera cosas como:
\begin{itemize}
\item \verb=print (int) argc=
\item \verb=print (char*) argv[0]=
\item \verb=print *(struct mi_estructura*) puntero=
\end{itemize}
\item Para inspeccionar memoria se usa el comando \verb=x/= así:
\begin{itemize}
\item \verb=x/x ADDR= imprime un entero apuntado por \verb=ADDR=
\item \verb=x/s ADDR= imprime un string que empieza en \verb=ADDR=
\item \verb=x/c ADDR= imprime un byte apuntado por \verb=ADDR=
\end{itemize}
\end{itemize}
\end{frame}

\begin{frame}[fragile]{Primera sesión de gdb - gdb shell}
\begin{lstlisting}
#include <stdio.h>
struct mi_estructura{
    int entero;
    float flotante;
    char cadena[20];
};
int main(){
    struct mi_estructura datos;
    memcpy(&datos, "\x39\x30\x00\x00\xcd\xcc\x8c\x3f"
                   "\x48\x6f\x6c\x61\x2c\x20\x6d\x75"
                   "\x6e\x64\x6f\x00",28);

    if (datos.entero + datos.flotante > 0.2)
	    printf("datos.cadena: %s\n",
	        datos.cadena);
}
\end{lstlisting}
\end{frame}
\begin{frame}[fragile]{Primera sesión de gdb - gdb shell}
\begin{itemize}
\item Le línea de comando de gdb esta paginada.
\item Si un comando imprime información de mas de una pagina se de tiene la ejecución
\item El tamaño de la página se define así:
\begin{verbatim}
(gdb) set height 100
\end{verbatim}
\item Toda la salida puede guardarse en un archivo de {\it log} con:
\begin{verbatim}
(gdb) set log on /tmp/archivo.log
\end{verbatim}
\item para dejar de {\it loggear}
\begin{verbatim}
(gdb) set log off
\end{verbatim}
\item El código fuente se puede listar con el comando \verb=list=
\item \verb=ENTER= a secas repite el comando anterior
\end{itemize}
\end{frame}

\begin{frame}[fragile]{Primera sesión de gdb - Expresiones automaticas}
\begin{itemize}
\item Se puede {\it seguir} el valor de una variable o expresión con el comando \verb-display-
\item \verb=display /fmt EXPR= agrega una expresión ala lista de expresiones automáticas
\item \verb=undisplay NUM= remueve cierta expresión de la lista
\item \verb=info display= muestra la lista de expresiones automáticas
\item Ejemplos:
\begin{itemize}
\item \verb=display /s miString=
\item \verb=display /d miEntero=
\item \verb=display *(struct MiStruct*) miStruct=
\end{itemize}
\end{itemize}
\end{frame}


\section{Ejecutando programas desde cero}

\begin{frame}[fragile]{Ejecutando programas en GDB}
\begin{itemize}
\item Para iniciar la depuración de un programa se usan: 
\begin{itemize}
\item \verb=run= ejecuta desde el comienzo
\item \verb=start= idem pero se detiene en \verb=main=
\end{itemize}
\item La traza de ejecución de un programa depende de la información proporcionada por el entorno 
\item Todo esto puede controlarse desde gdb:
\begin{itemize}
\item Los {\it argumentos} de la línea de comando
\item Las {\it variables de entorno} {\small ver \verb=env=}
\item El {\it directorio de trabajo}
\item La {\it salida y entrada estándard}
\end{itemize}
\end{itemize}
\end{frame}

\begin{frame}[fragile]{Los argumentos}
\begin{itemize}
\item Los argumentos se pueden asignar previamente así:
\begin{verbatim}
(gdb) set args arg1 arg2 arg3
(gdb) run
\end{verbatim}

\item o se pueden pasar normalmente a \verb=run= o \verb=start=
\begin{verbatim}
(gdb)run arg1 arg2 arg3
\end{verbatim}
(si no se especifica nada se usa lo seteado anteriormente)
\item Para verlos: \verb=show args=
\item Si se puede son procesados por el {\tt shell}. Es decir {\tt *} se reemplazará por los
nombres de archivo del directorio actual.
\end{itemize}
\end{frame}

\section{Controlando el entorno}

\begin{frame}[fragile]{Los argumentos - Ejemplito}
\begin{lstlisting}
int
main (int argc, char *argv[])
{
  int i;
  /* Esperamos al menos 1 argumento */
  assert (argc==2);
  /* En el primer argumento esperamos un numero */
  sscanf(argv[1],"%u",(unsigned int*)&i);
  /* un numero menor que 100*/
  assert(i<100 && i>=0);
  printf("El parametro %d esta entre 0 y 100\n",i);
  return 0;
}
\end{lstlisting}
\end{frame}

\begin{frame}[fragile]{Variables de entorno y directorio}
\begin{itemize}
\item Las variables de entorno se setean así
\begin{verbatim}
(gdb) set environment VARIABLE="Este es el valor"
\end{verbatim}
\item Se pueden revisar así:
\begin{verbatim}
(gdb) show environment VARIABLE
\end{verbatim}
\item Y borrar así:
\begin{verbatim}
(gdb) unset environment VARIABLE
\end{verbatim}
\item El directorio actual o de trabajo se setea con \verb=cd= y se revisa con \verb=pwd= directamente desde el gdb
\end{itemize}
\end{frame}


\begin{frame}[fragile]{Variables de entorno y directorio - Ejemplito}
\begin{lstlisting}
int
main (int argc, char *argv[])
{
  char cwd[1024];
  char * variable;
  variable = getenv("VARIABLE");
  /* La VARIABLE de entorno debe existir */
  assert (variable != NULL);
  printf("VARIABLE=%s\n",variable);
  /* El directorio actual debe ser /etc */
  assert(getcwd(cwd, sizeof(cwd)) != NULL);
  assert(strcmp("/etc",cwd)==0);
  printf("El directorio actual es %s\n",cwd);
  return 0;
}
\end{lstlisting}
\end{frame}


\begin{frame}[fragile]{Entrada y salida estándard }
\begin{itemize}
\item La entrada y salida estándard se puede direccionar al igual que en bash
\item Funciona con el con \verb=run= y \verb=start= así:
\begin{verbatim}
(gdb) run < archivo.in >archivo.out
\end{verbatim}
\end{itemize}

\end{frame}

\begin{frame}[fragile]{Entrada y salida estándard - Ejemplito}
\begin{lstlisting}
int main (int argc, char *argv[]) 
{
  FILE *f;
  char bufferA[1024];
  char bufferB[1024];  
  char cwd[1024];
  /* El directorio actual debe ser /etc */
  assert (getcwd (cwd, sizeof (cwd)) != NULL);
  assert (strcmp ("/etc", cwd) == 0);
  printf ("El directorio actual es %s\n", cwd);
  f = fopen("passwd","r");
  fread(bufferA,1,1024,f);
  fread(bufferB,1,1024,stdin);
  assert(memcmp(bufferA,bufferB,1024)==0);
  printf("stdin es identico a /etc/passwd\n");  
  return 0;
}
\end{lstlisting}
\end{frame}

\section{Depurando programas en ejecución}

\begin{frame}[fragile]{Depurando programas en ejecución}
\begin{itemize}
\item Se usa \verb=attach=, \verb=dettach= y \verb=kill= para depurar un programa en ejecución
\item Inmediatamente después de un \verb=attach= el programa queda suspendido
\item El programa siendo depurado puede interrumpirse con \verb=CTRL+C= y reanudarse con \verb=continue=
\item La memoria o variables del programa se inspeccionan con \verb=print= o \verb=x= así:
\begin{itemize}
\item \verb=(gdb) print flag=
\item \verb=(gdb) x/s tema=
\end{itemize}
\item Para modificar variables se puede usar \verb=set= o incluso \verb=print= así:
\begin{itemize}
\item \verb#(gdb) set flag=0#
\item \verb#(gdb) p strcpy(tema,"Me puedo debugear")#
\end{itemize}
\end{itemize}
\end{frame}


\begin{frame}[fragile]{Depurando programas en ejecución - Ejemplito}
\begin{lstlisting}
int
main (int argc, char *argv[])
{
  char tema[]="Me puedo programar";
  char banda[]="Virus";
  int flag=1;
  printf("Hola mi pid es %d\n", getpid());
  while(flag);

  printf("Mi tema preferido es '%s' de la banda"\
                            "'%s'.\n",tema,banda);
  printf("Link http://www.youtube.com/"\
                        "watch?v=r23-Y6ElWV4.\n");
  return 0;
}
\end{lstlisting}
\end{frame}

\section{Depurando programas con hilos o forks}

\begin{frame}[fragile]{Depurando programas con hilos}
GDB provee algunas facilidades para depurar hilos
\begin{itemize}
\item Notifica automáticamente la creación de hilos
\item \verb=info threads= lista los hilos existentes
\item En cada momento se está examinando el contexto de un hilo determinado
\item \verb=thread NUM= cambia el hilo actual
\item \verb=thread apply all= ejecuta un comando GDB en todos los hilos. Por ej.
\begin{verbatim}
(gdb) thread apply all p flag
(gdb) thread apply all p flag=0
\end{verbatim}
\item Para ejecutar un solo hilo por vez usar algo así:
\begin{verbatim}
(gdb) set scheduler-locking on
\end{verbatim}
\item Se puede consultar con \verb=show scheduler-locking=
\end{itemize}
\end{frame}


\begin{frame}[fragile]{Depurando programas con hilos - Ejemplito}
\begin{lstlisting}
void * inc (void *ptr){
  int flag = 1;
  /* Loop loco */
  while (flag==1)
    ++(*(int *) ptr);
  printf ("Salio despues de %d iteraciones\n",*(int *) ptr);
  return NULL;
}
int main (int argc, char * argv[]){
  pthread_t thread[2];
  int x = 0, y = 0;
  pthread_create (&thread[0], NULL, inc, &x);
  pthread_create (&thread[1], NULL, inc, &y);
  pthread_join(thread[1], NULL);
  pthread_join(thread[0], NULL);
  return 0;
}
\end{lstlisting}
\end{frame}

\begin{frame}[fragile]{Depurando programas con forks}
Para controlar el comportamiento de gdb en tiempo de \verb=fork= se usa
\verb=follow-fork-mode=
\begin{itemize}
\item \verb#set follow-fork-mode child# instruye al gdb a {\it seguir} siempre al proceso hijo
\item \verb#set follow-fork-mode parent# instruye al gdb a {\it quedarse} en el proceso hijo
\item \verb#show follow-fork-mode# muestra el estado actual de la variable
\end{itemize}
\end{frame}

\begin{frame}[fragile]{Depurando programas con forks - Ejemplito}
\begin{lstlisting}
#include <sys/types.h>
#include <unistd.h>
#include <stdio.h>
int
main (int argc, char *argv[])
{
  pid_t pid = fork ();
  if (!pid)
      printf ("Proceso hijo: PID %d\n", getpid());
  else
      printf ("Proceso padre: PID %d\n", getpid());
  return 0;
}
\end{lstlisting}
\end{frame}

\section{Depurando programas con señales}

\begin{frame}[fragile]{Depurando programas con señales}
\begin{itemize}
\item Las {\it signals} son eventos asíncronos provistos por el sistema operativo. (ver \verb=man 7 signal=)
\item \verb=info signals = imprime la lista de todas las señales y cómo gdb esta configurado para manejarlas
\item Cada señal puede ser interpretada por gdb de diferentes maneras, este comportamiento se configura con:
\begin{verbatim}
(gdb)handle SIGNAL KEYWORD
\end{verbatim}
... donde \verb=KEYWORD= es uno de:
\begin{itemize}
\item \verb=stop= y \verb=nostop= detiene o no la ejecución
\item \verb=print= y \verb=noprint= imprime o no un mensaje informativo
\item \verb=pass= y \verb=nopass= pasa o nopasa la señal al programa (es decir si ejecuta o no el manejador asociado en el programa)
\end{itemize}
\end{itemize}
\end{frame}


\begin{frame}[fragile]{Depurando programas con Señales - Ejemplito}
\begin{lstlisting}
int cont = 0; int flag = 1;
void handler (int sig) {
  signal (SIGALRM, SIG_IGN);
  if(flag){
    printf ("SIGALRM exiting! :(");
    exit(1);
  }
  alarm (2);
  signal (SIGALRM, handler);
}
void main (int argc, char *argv[]){
  signal (SIGALRM, handler);
  alarm (2);
  while (cont<500){
      sleep(1);
      cont++;
  }
  printf("Ganaste!\n");
}
\end{lstlisting}
\end{frame}



\section{Interrumpiendo la ejecución}

\begin{frame}[fragile]{Interrumpiendo la ejecución - Motivación}
\begin{lstlisting}
int readSUM(istream& in){
  char buffer[256];
  int i, co;
  assert(in.read (buffer,256).good());
  for(i=0,co=0;i<in.gcount();i++)
    co+=buffer[i];
  return co;
}
int main (int argc, char* argv[]) {
  int checksum;
  checksum = readSUM(cin);
  assert(checksum > 1000 && checksum < 16640);
  cout << "El checksum es: " << checksum <<endl;
  return 0;
}
\end{lstlisting}
\end{frame}


\begin{frame}[fragile]{Interrumpiendo la ejecución}
\begin{itemize}
\item El propósito principal de usar un depurador es poder detener el procesamiento en puntos de interés (o analizar cuando se está en algún crash/deadlock)
\item GDB provee tres formas de detener la ejecución en lugares interesantes:
\begin{itemize}
\item {\it breakpoints}, detienen el programa en un lugar o punto del código
\item {\it wathchpoints}, detienen el programa cuando cierto dato se accede o modifica
\item {\it catchpoints}, detiene el programa cuando se da una excepción de C++ o se carga un módulo o biblioteca
\end{itemize}
\item gdb asigna un número a cada {\it breakpoint}, {\it watchpoint} y {\it catchpoint} que se crea
\item ese número se usa como referencia desde otros comandos, sirve para la habilitación, deshabilitación y borrado de los mismos
\end{itemize}
\end{frame}

\begin{frame}[fragile]{Interrumpiendo la ejecución - Breakpoins}
\begin{itemize}
\item Los {\it breakpoints} se pueden setear de muchas maneras.
\begin{itemize}
\item \verb=break nombre_de_funcion= pone un breakpoint al comienzo de una función 
\item \verb=break archivo.c:línea= pone un breakpoint en el número de línea especificado 
\item \verb=break= a secas pone un breakpoint en la posición actual
\item \verb=rbreak regExp= pone un breakpoint en todas la funciones que cumplan con la expresión regular regExp. Ej \verb=rbreak .*alloc.*=

\end{itemize}
\end{itemize}
\end{frame}

\begin{frame}[fragile]{Interrumpiendo la ejecución - Watchpoints}
\begin{itemize}
\item Los {\it watchpoints} monitorean el estado de ciertas partes de la memoria
\begin{itemize}
\item \verb=watch EXPR= detiene la ejecución cuando \verb=EXPR= es modificada
\item \verb=rwatch EXPR= detiene la ejecución simplemente cuando \verb=EXPR= es leída
\item \verb=awatch EXPR= detiene la ejecución cuando \verb=EXPR= es modificada y/o leída
\end{itemize}
\item \verb=delete=, \verb=disable= y \verb=enable= hacen lo esperado
\item \verb=info watchpoints= muestra la lista de watchpoints
\end{itemize}
\end{frame}

\begin{frame}[fragile]{Interrumpiendo la ejecución - Catchpoints}
\begin{itemize}
\item Se pueden utilizar los {\it catchpoints} para detener la ejecución en ciertos eventos del programa. Se usan así:
\begin{verbatim} 
(gdb) catch EVENT
\end{verbatim} 
donde \verb=EVENT= es alguno de los siguientes eventos:
\begin{itemize}
\item \verb=throw= cuando se lanza una excepción de C++
\item \verb=catch= cuando se maneja una excepción de C++
\item \verb=load LIBNAME= cuando se carga \verb=LIBNAME=  a la memoria del proceso
\item \verb=unload LIBNAME=  cuando se descarga \verb=LIBNAME= de la memoria del proceso
\end{itemize}
\end{itemize}
\end{frame}

\begin{frame}[fragile]{Interrumpiendo la ejecución - Administración}
\begin{itemize}

\item \verb=delete NBREAK= borra el breakpoint
\item \verb=disable NBREAK= lo deshabilita temporalmente
\item \verb=enable NBREAK=  lo vuelve a habilitar 
\item \verb=ignore NBREAK NUM= ignorara \verb=NUM= veces el breakpoint \verb=NBREAK=  
\item \verb=info breakpoints= (o \verb=watchpoints=, o \verb=catchpoints=) muestra el estado de los breakpoints (o watchpoints, o catchpoints)
\end{itemize}
\end{frame}

\begin{frame}[fragile]{Interrumpiendo la ejecución - Ejemplito}
\begin{lstlisting}
class NewDelete{
    char local[256];
    char *c;
    size_t len;
    int error;
public:
    NewDelete (size_t size) {
       try{
            c = new char[size];
            len = size;
       }catch (...) {
            cout << "No hay suficiente memoria" <<endl;
            globalerror = error = 1;
       }
       error = 0;
    }

    ~NewDelete(){ delete c; }
};
\end{lstlisting}
\end{frame}

\begin{frame}[fragile]{Interrumpiendo la ejecución - Ejemplito (..cont)}
\begin{lstlisting}
void
newdelete(size_t size){
NewDelete nd(size);
}
int main () {
  int checksum;
  checksum = readSUM(cin);
  newdelete(checksum);
  cout << "El checksum es: " << checksum <<endl;
  return 0;
}
\end{lstlisting}
\end{frame}


\begin{frame}[fragile]{Interrumpiendo la ejecución - Efectos secundarios}
\begin{itemize}

\item Se puede asignar condiciones a casi cualquier cosa de esta manera:
\begin{verbatim}
(gdb) condition NBREAK flag==1
\end{verbatim}

\item y se remueven así:
\begin{verbatim}
(gdb) condition NBREAK
\end{verbatim}

\item Tambien se puede ejecutar una lista de comandos en cada breakpoint!
\begin{verbatim}
(gdb) command NBREAK
print "BREAKPOINT!"
continue
end
\end{verbatim}

\item y se quita asignándole la lista vacía de comandos:
\begin{verbatim}
(gdb) command NBREAK
end
\end{verbatim}

\end{itemize}
\end{frame}
\section{Continuando la ejecución}
\begin{frame}[fragile]{Continuando la ejecución}
\begin{itemize}
\item La ejecución puede reanudarse con \verb=continue=
\item \verb=next= ejecuta la próxima línea
\item \verb=stepi= ejecuta la próxima línea metiéndose dentro de las funciones
\item \verb=finnish= ejecuta hasta salir de la función actual
\item \verb=until = ejecuta hasta pasar número de línea especificado. Optimo para salir de bucles!
\item \verb=jump NUMLINEA= continua ejecutando desde la linea de codigo especificada
\item \verb=return VALOR= ignora el comportamiento restante de la funcion actual y retorna \verb=VALOR= a la llamante
\end{itemize}
\end{frame}

\section{Analizando la pila}

\begin{frame}[fragile]{Analizando la pila}
\begin{itemize}
\item OK, el programa se detuvo. Pero como llegué ahí?
\item Cada vez que se realiza una llamada a procedimiento se {\it apila} información
para poder retornar del mismo
\item Esto se denomina marco de activación o {\it stack frame}
\item GDB asigna un número a cada marco de activación anidado y se listan así:
\begin{verbatim}
(gdb) backtrace
\end{verbatim}
\item Sólo pueden accederse a las variables locales del marco actual
\item Se puede navegar o cambiar de marco con:
\begin{verbatim}
(gdb) frame N
\end{verbatim}
\item O iterativamente usando \verb=up= y \verb=down=
\item \verb=info args= e \verb=info locals= muestran los argumentos y las variables locales del marco actual 
\item \verb=info catch= lista los manejadores de excepciones activos
\end{itemize}
\end{frame}

\begin{frame}[fragile]{Analizando la pila - Ejemplito}
\begin{lstlisting}
#include <stdio.h>
int fib(int n)
{
if(n==1 || n==0)
    return n;
else
    return fib(n-1)+fib(n-2);
}

int main()
{
    printf("Fibonacci(%d)= %d\n\n ",10,fib(10));
    return 0;
}
\end{lstlisting}
\end{frame}


\section{Comandos definidos por el usuario}
\begin{frame}[fragile]{Comandos definidos por el usuario}
\begin{itemize}
\item GDB permite definir comandos personalizados
\item Son una lista de comandos comunes de gdb empaquetados y con nombre
\item Pueden tomar argumentos
\item Tienen esta pinta
\begin{verbatim}
define suma
print $arg0 $arg1 $arg2
end
\end{verbatim}
\item y se usan asi:
\begin{verbatim}
(gdb) suma 1 2 3
$1 = 6
\end{verbatim}
\end{itemize}
\end{frame}

\begin{frame}[fragile]{Comandos definidos por el usuario}
\begin{itemize}
\item Tiene \verb=if=, \verb=while= y variables auxiliares como \verb=$i= y \verb=$aux=
\begin{verbatim}
(gdb) set $i=0
(gdb) while $i<100
set $i=$i+1
if $i%3==0
  printf "%d es multiple de 3\n", $i
  end
end
\end{verbatim}
\item Los {\it scopes} se cierran con \verb=end=
\item Ideales para crear una biblioteca de funciones de conveniencia en \verb=.gdbinit= 
\item Probar esto y luego depurar un programa (usando \verb=start=):
\begin{verbatim}
define hook-start
printf "STAAAAART!"
end
\end{verbatim}
\end{itemize}
\end{frame}

\section{Crashes y coredumps}
\begin{frame}[fragile]{Crashes y coredumps}
\begin{quotation}
In computing, a core dump, more properly a memory dump or storage dump,
consists of the recorded state of the working memory of a computer
program at a specific time, generally when the program has terminated
abnormally 
\end{quotation}

\begin{itemize}
\item la generación o no de \verb=coredumps= se controla con el comando \verb=ulimit=
\begin{verbatim}
$ ulimit -c unlimited
\end{verbatim}
\item tambien ver \verb=man core= para información detallada
\item algunas senales generan por defecto coredumps
\begin{itemize}
\item Program terminated with signal 4, Illegal instruction.
\item Program terminated with signal 6, Aborted.
\item Program terminated with signal 8, Arithmetic exception.
\item Program terminated with signal 11, Segmentation fault.
\end{itemize}
\end{itemize}
\end{frame}


\begin{frame}[fragile]{Crashes y coredumps - División por cero}
\begin{lstlisting}
void
sigpfe ()
{
    int a = 1;
    int b = 0;
    float c;
    c = a/b;
    c++;
}
\end{lstlisting}
\end{frame}

\begin{frame}[fragile]{Crashes y coredumps - Fallo de pagina de memoria}
\begin{lstlisting}
void
sigsegv ()
{
    *(int*)2340=1000;
}
\end{lstlisting}
\end{frame}

\begin{frame}[fragile]{Crashes y coredumps - Abort!}
\begin{lstlisting}
void
sigabort ()
{
    char * p = (char*)malloc(1);
    free(p);
    free(p);
    //abort();
}
\end{lstlisting}
\end{frame}

\begin{frame}[fragile]{Crashes y coredumps - Instruccion ilegal}
\begin{lstlisting}
void
sigill ()
{
    asm ( ".byte -2\n" );
}
\end{lstlisting}
En otros OSs tambien aparece \verb=SIGBUS=
\end{frame}




\section{Depurando remotamente}
\begin{frame}[fragile]{Depurando remotamente - gdbserver}
\begin{itemize}
\item \verb=gdbserver= es un depurador reducido que puede controlarse via gdb
\item \verb=gdb= y \verb=gdbserver= se comunican por TCP o una linea serial
\item Una vez establecida la comunicación GDB se comporta normalmente
\item El server se ejecuta así:
\begin{verbatim}
$ gdbserver host:2345 programa param1 param2
\end{verbatim}
\item Y desde el gdb remoto te conectas así:
\begin{verbatim}
(gdb) target remote host:2345 
\end{verbatim}
\end{itemize}
\begin{quotation}
gdbserver is not a complete replacement for the debugging stubs, because it requires essentially the same operating-system facilities that GDB itself does. In fact, a system that can run gdbserver to connect to a remote GDB could also run GDB locally!
\end{quotation}
\end{frame}

\begin{frame}[fragile]{Depurando remotamente - otra opción?}
\begin{itemize}
\item Poner un GDB completo en el target
\item Conetarse por ssh/telnet
\item Con \verb=show directory= se listan los caminos donde gdb busca los archivos de código fuente
\item Con \verb=directory CAMINO= se agregan CAMINOS donde GDB buscara el código
\item Montar un \verb=nfs= en el target con el repositorio de fuentes en uso
\item Probarlo con \verb=list=
\end{itemize}
\end{frame}


\section{Referencias}
\begin{frame}{Referencias}
\begin{itemize}
\item \url{http://gdb.gnu.org/}
\item \url{http://www.acsu.buffalo.edu/~charngda/gdb.html}
\item \url{http://www.ibm.com/developerworks/aix/library/au-gdb.html}
\item \url{http://sourceware.org/gdb/onlinedocs/gdb/}
\item \url{mailto:felipe.andres.manzano@gmail.com}
\end{itemize}
\end{frame}


{\usebackgroundtemplate{\includegraphics[width=\paperwidth,height=\paperheight]{machinalis_b}}
\begin{frame}{}\end{frame}
}

\end{document}

